Very lightweight testing framework for F\+O\+R\+T\+R\+AN, written entirely in F\+O\+R\+T\+R\+AN. Supports basic assertions and running of test sets.

\subsection*{Installation}

Clone the \hyperlink{namespacefort__test}{fort\+\_\+test} repo into your project directory

\subsection*{Usage}

Simply type


\begin{DoxyCode}
use fort\_tests
\end{DoxyCode}


At the top of your file. You can make a \hyperlink{runtests_8f90}{runtests.\+f90} file that can compile to a runtests executable to perform all of your tests without compiling the main program you\textquotesingle{}re working on. I have included a basic makefile and example \hyperlink{runtests_8f90}{runtests.\+f90} file to show how you could do this

To make a test file, first declare at least one {\ttfamily Test\+Set} structure and (optionally) a named array of {\ttfamily Test\+Set}s.


\begin{DoxyCode}
\textcolor{keywordtype}{type}(testset):: testset\_1, testset\_2
\textcolor{keywordtype}{type}(testset), \textcolor{keywordtype}{dimension(:)}, \textcolor{keywordtype}{allocatable}:: my\_testsets
\end{DoxyCode}


Next, use the {\ttfamily new\+\_\+testset} constructor to build (and optionally name) your testsets and fill them with tests. To see the results of your tests, call the {\ttfamily print\+\_\+results} subroutine, which takes an array of {\ttfamily Test\+Sets} as an argument


\begin{DoxyCode}
testset\_1 = new\_testset(    &
  (/                        &
    assert(.true.),         &
    assert\_eq(2.0, 1.0 + 1.0),  &
    assert\_neq(2.0, 2.0 + 2.0), &
    assert\_approx(4.d0, 4.d0 + 10.d0*epsilon(4.d0)) &
  /),                       &
  name = \textcolor{stringliteral}{"Sample test set"}  &
)

my\_testsets = (/ testset\_1 /)
\textcolor{keyword}{call }print\_results(my\_testsets)
\end{DoxyCode}


All of our tests are self evidently correct, so we should get the following output, all nicely colored\+:



Lets make some tests that fail now. We\textquotesingle{}ll copy our first testset and make all of the tests fail.


\begin{DoxyCode}
testset\_2 = new\_testset(    &
  (/                        &
    assert(.false.),         &
    assert\_eq(3.0, 1.0 + 1.0),  &
    assert\_neq(4.0, 2.0 + 2.0), &
    assert\_approx(3.d0, 4.d0 + 10.d0*epsilon(4.d0)) &
  /),                       &
  name = \textcolor{stringliteral}{"Failing test set"}  &
)

my\_testsets = (/ testset\_1, testset\_2 /)
\textcolor{keyword}{call }print\_results(my\_testsets)
\end{DoxyCode}




The program provides minimal but helpful messages here. It doesn\textquotesingle{}t have the ability to read the line of sourcecode that produced the error so we can only use whatever arguments you pass in. The message for a failing test using the basic \textquotesingle{}assert\textquotesingle{} function will always be pretty sparse, but the test numbers will help you figure out which line of code is failing. Use the other assertion functions if you want more detailed readout.

That about concludes the basic tutorial. More documentation will be coming in the future! Please let me know if you have any questions or if you have functionality you\textquotesingle{}d like included. 